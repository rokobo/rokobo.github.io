\documentclass[a4paper,10pt]{article}
\usepackage[utf8]{inputenc}
\usepackage{enumitem}
\usepackage[colorlinks=true, urlcolor=blue, linkcolor=blue]{hyperref}
\usepackage{ulem}
\newcommand{\ulink}[2]{\href{#1}{\underline{#2}}}
\usepackage{textcomp}
\usepackage{geometry}
\geometry{a4paper, margin=7mm}

\title{Pedro Ramos Kobori Portfólio}
\begin{document}
\date{}
\author{Pedro Ramos Kobori}

\noindent
\begin{minipage}[t]{0.5\textwidth}
  \begin{flushleft}
    \textbf{\Large Pedro Ramos Kobori}
  \end{flushleft}
\end{minipage}
\begin{minipage}[t]{0.5\textwidth}
  \begin{flushright}
    \textbf{\Large Software developer}
  \end{flushright}
\end{minipage}

\hrule

\vspace{2mm}
\noindent
{
  \centering
  Campinas, São Paulo, Brazil \textbar{}
  pedro.kobori@gmail.com \textbar{}
  \ulink{https://github.com/rokobo}{Github} \textbar{}
  \ulink{https://www.linkedin.com/in/pedrokobori/}{LinkedIn} \textbar{}
  \ulink{https://rokobo.github.io}{Portfólio}
  \par
}

\section*{Skills}
\hrule
\vspace{2mm}
\begin{itemize}[itemsep=0pt]
  \item \textbf{Linguagens}: Python, SQL, Excel, LaTeX, RegEx, Ruby, Julia, Assembly, SML, Racket, UML.
  \item \textbf{Tecnologias}: NumPy, Pandas, TensorFlow, Docker, Pytest, GitHub, Spark, Neo4j, Splunk, Knime, Apache Airflow.
  \item \textbf{Web Dev}: Plotly Dash, HTML, CSS, GitHub Actions, Javascript, Matplotlib.
\end{itemize}

\section*{Projetos}
\hrule
\vspace{2mm}
\textbf{Big data specialization}, \ulink{https://github.com/rokobo/Big-Data-Specialization}{Código do Projeto}.
\begin{itemize}[itemsep=0pt]
  \item Realizei exploração de dados com Splunk para análise de desempenho de vendas no mercado de itens.
  \item Criei um fluxo de trabalho de modelo de classificação com Knime para atribuir rótulos a usuários com altos gastos.
  \item Utilizei cluster K-means com Spark (PySpark) para análise de segmento de usuário.
  \item Usei modelagem de dados de chat com Neo4j para recomendar campanhas de marketing direcionadas.
\end{itemize}
\textbf{Productivity autotracker}, \ulink{https://github.com/rokobo/Productivity-autotracker}{Código do projeto}.
\begin{itemize}[itemsep=0pt]
  \item Implementei rastreamento de atividades e categorização de eventos em Python para determinar produtividade do usuário.
  \item Bancos de dados SQL usados para acompanhar dados, tendências, configurações e marcos do usuário.
  \item Fiz a interface e configuração usando Plotly Dash, com configurações de usuário facilmente configuráveis.
  \item Pipelines de dados e backups feitos usando Apache Airflow.
\end{itemize}
\textbf{Machine learning specialization}, \ulink{https://github.com/rokobo/Machine-learning-specialization}{Código do projeto}.
\begin{itemize}[itemsep=0pt]
  \item Criação de redes neurais para classificação de imagens, filtragem colaborativa para recomendação de filmes.
  \item Criação de detecção de anomalias para operações de servidor, aprendizado de reforço para módulo lunar.
  \item Criado clustering para compactação de imagens, árvores de decisão para reconhecimento de cogumelos.
  \item Executei análise de modelo, engenharia de dados e visualização de dados para vários problemas de ML do mundo real.
  \item Tecnologias: NumPy, TensorFlow, Keras, Scikit-Learn, Matplotlib, Pandas, OpenAI Gym.
\end{itemize}
\textbf{Hack computer design and game implementation}, \ulink{https://github.com/rokobo/From-Nand-gates-to-Tetris-implementation}{Código do projeto}.
\begin{itemize}[itemsep=0pt]
  \item A partir de uma abstração de porta NAND e algumas ferramentas de emulação, implementei o software Hack.
  \item Implementei a arquitetura teórica de hardware usando HDL.
  \item Implementei um assembler, para transformar arquivos assembly em arquivos Hack (código binário para a plataforma Hack).
  \item Implementei um tradutor, traduz o código VM (operações baseadas em pilha) em assembly.
  \item Implementei um compilador para compilar arquivos Jack em código VM.
  \item Finalmente, implementei um Jack OS e um jogo Jack.
\end{itemize}
\textbf{DNA sequence analysis using alignment and scoring matrices}, \ulink{https://github.com/rokobo/DNA-Sequence-Analysis}{Código do Projeto}.
\begin{itemize}[itemsep=0pt]
  \item Implementei funções para criar matrizes de alinhamento e pontuação para quantificar similaridade em 2 sequências de DNA.
  \item Realizei testes estatísticos de hipóteses com escores Z de múltiplos alinhamentos locais com uma sequência aleatória.
  \item Funções implementadas para quantificar a dissimilaridade entre 2 strings para verificação ortográfica usando distâncias de edição.
\end{itemize}
\textbf{Computer network resilience analysis}, \ulink{https://github.com/rokobo/Computer-Network-Resilience-Analysis}{Código do projeto}.
\begin{itemize}[itemsep=0pt]
  \item Analisei a conectividade de uma rede de computadores desabilitando aleatoriamente computadores na rede.
  \item Comparei a resiliência de um gráfico fornecido e gráficos ER e UPA gerados aleatoriamente (variedade de gráfico DPA).
  \item Implementado e comparado (complexidade de tempo) 3 algoritmos para remoção de computadores da rede.
  \item Analisei como o maior componente conectado mudou dependendo do algoritmo de criação do gráfico.
\end{itemize}

\section*{Educação}
\hrule
\vspace{2mm}
\begin{itemize}[itemsep=0pt]
  \item Atualmente estudando pelo
        \ulink{https://github.com/ossu/computer-science}{Open Source Society University (OSSU)}
        caminho de ciência da computação. Veja as certificações obtidas em meu portfólio. O currículo do OSSU
        é uma educação completa em ciência da computação usando materiais online. Ele é projetado
        de acordo com as diretrizes curriculares para programas de graduação em ciência de domputação, feito pela IEEE.
  \item Português nativo, inglês fluente: Certificado Cambridge Advanced English Level 2 em ESOL Internacional.
\end{itemize}

\end{document}
