\documentclass[a4paper,10pt]{article}
\usepackage[utf8]{inputenc}
\usepackage{enumitem}
\usepackage[colorlinks=true, urlcolor=blue, linkcolor=blue]{hyperref}
\usepackage{ulem}
\newcommand{\ulink}[2]{\href{#1}{\underline{#2}}}
\usepackage{textcomp}
\usepackage{geometry}
\geometry{a4paper, margin=1cm}

\title{Pedro Ramos Kobori's Portfolio}
\begin{document}
\date{}
\author{Pedro Ramos Kobori}

\noindent
\begin{minipage}[t]{0.5\textwidth}
  \begin{flushleft}
    \textbf{\Large Pedro Ramos Kobori}
  \end{flushleft}
\end{minipage}
\begin{minipage}[t]{0.5\textwidth}
  \begin{flushright}
    \textbf{\Large Software developer}
  \end{flushright}
\end{minipage}

\hrule

\vspace{2mm}
\noindent
{
  \centering
  Campinas, São Paulo, Brazil \textbar{}
  pedro.kobori@gmail.com \textbar{}
  \ulink{https://github.com/rokobo}{Github} \textbar{}
  \ulink{https://www.linkedin.com/in/pedrokobori/}{LinkedIn} \textbar{}
  \ulink{https://rokobo.github.io}{Portfolio}
  \par
}

\section*{Skills}
\hrule
\vspace{2mm}
\begin{itemize}[itemsep=0pt]
  \item \textbf{Programming}: Python, SQL, Excel, LaTeX, RegEx, Ruby, Julia, Assembly, SML, Racket, UML, Java.
  \item \textbf{Technologies}: NumPy, Pandas, TensorFlow, Docker, Pytest.
  \item \textbf{Web Development}: Plotly Dash, HTML, CSS, GitHub Actions, Javascript.
\end{itemize}

\section*{Projetos}
\hrule
\vspace{2mm}
\textbf{Autotracker de Produtividade}, \ulink{https://github.com/rokobo/Productivity-autotracker}{Código do Projeto}.
\begin{itemize}[itemsep=0pt]
  \item Implementei a lógica de rastreamento de atividades e categorização de eventos em Python para determinar a produtividade do usuário.
  \item Utilizei bancos de dados SQL para manter o controle dos dados do usuário, tendências, configurações e conquitas.
  \item Criei a interface e configuração usando Plotly Dash, com configurações de usuário facilmente configuráveis.
\end{itemize}
\textbf{Design de computador Hack e implementação de jogo}, \ulink{https://github.com/rokobo/From-Nand-gates-to-Tetris-implementation}{Código do Projeto}.
\begin{itemize}[itemsep=0pt]
  \item A partir de uma abstração de porta NAND e ferramentas de emulação, implementei o software do computador Hack.
  \item Implementei a arquitetura de hardware teórica usando HDL.
  \item Implementei um assembler, para transformar arquivos de assembly em arquivos Hack (código binário para a plataforma Hack).
  \item Implementei um tradutor, que traduz código VM (operações baseadas em pilha) em assembly.
  \item Implementei um compilador, para compilar arquivos Jack em código VM.
  \item Por fim, implementei um OS Jack e um jogo Jack.
\end{itemize}
\textbf{Agendador de eventos}, \ulink{https://github.com/rokobo/Event-Scheduler}{Código do Projeto}.
\begin{itemize}[itemsep=0pt]
  \item Implementei um agendador de eventos usando um algoritmo de backtracking recursivo em uma aplicação web Dash.
  \item A GUI foi implementada usando CSS e o framework Dash na linguagem Julia.
  \item Testes unitários foram realizados em todas as funcionalidades e funções especiais.
\end{itemize}
\textbf{Análise de sequência de DNA usando matrizes de alinhamento e pontuação}, \ulink{https://github.com/rokobo/DNA-Sequence-Analysis}{Código do Projeto}.
\begin{itemize}[itemsep=0pt]
  \item Implementei funções para criar matrizes de alinhamento e matrizes de pontuação para quantificar a semelhança entre 2 sequências de DNA.
  \item Realizei testes de hipótese estatísticos com Z-scores de múltiplos alinhamentos locais com uma sequência aleatória.
  \item Implementei funções para quantificar a dissimilaridade entre 2 strings para verificação ortográfica usando edit distance.
\end{itemize}
\textbf{Análise da resiliência de redes de computadores}, \ulink{https://github.com/rokobo/Computer-Network-Resilience-Analysis}{Código do Projeto}.
\begin{itemize}[itemsep=0pt]
  \item Analisei a conectividade de uma rede de computadores desativando aleatoriamente computadores na rede.
  \item Comparei a resiliência de um gráfico fornecido e gráficos ER e UPA gerados aleatoriamente (variedade de gráfico DPA).
  \item Implementei e comparei (em termos de complexidade de tempo) 3 algoritmos para remover computadores da rede.
  \item Analisei como o maior componente conectado mudou dependendo do algoritmo de criação do gráfico.
\end{itemize}
\textbf{Intérprete de linguagem de objetos bidimensionais}, \ulink{https://github.com/rokobo/2D-objects-language}{Código do Projeto}.
\begin{itemize}[itemsep=0pt]
  \item Implementei um intérprete para uma linguagem fictícia criada para manipular objetos bidimensionais.
  \item O intérprete foi implementado em Ruby (OOP) e SML (FP) para estudar as diferenças de paradigma.
  \item Utilizei despacho duplo, despacho dinâmico, correspondência de padrões, herança e geometria.
\end{itemize}

\section*{Educação}
\hrule
\vspace{2mm}
\begin{itemize}[itemsep=0pt]
  \item Atualmente estudando pelo
        \ulink{https://github.com/ossu/computer-science}{Open Source Society University (OSSU)}
        caminho de ciência da computação. Veja as certificações obtidas em meu portfólio. O currículo do OSSU
        é uma educação completa em ciência da computação usando materiais online. Ele é projetado
        de acordo com as diretrizes curriculares para programas de graduação em ciência de domputação, feito pela IEEE.

  \item Português nativo, inglês fluente: Certificado Cambridge Advanced English Level 2
        em ESOL Internacional.
\end{itemize}

\end{document}
