\documentclass[a4paper,10pt]{article}
\usepackage[utf8]{inputenc}
\usepackage{enumitem}
\usepackage[colorlinks=true, urlcolor=blue, linkcolor=blue]{hyperref}
\usepackage{ulem}
\newcommand{\ulink}[2]{\href{#1}{\underline{#2}}}
\usepackage{textcomp}
\usepackage{geometry}
\geometry{a4paper, margin=7mm}

\title{Pedro Ramos Kobori's Portfolio}
\begin{document}
\date{}
\author{Pedro Ramos Kobori}

\noindent
\begin{minipage}[t]{0.5\textwidth}
  \begin{flushleft}
    \textbf{\Large Pedro Ramos Kobori}
  \end{flushleft}
\end{minipage}
\begin{minipage}[t]{0.5\textwidth}
  \begin{flushright}
    \textbf{\Large Software developer}
  \end{flushright}
\end{minipage}

\hrule

\vspace{2mm}
\noindent
{
  \centering
  Campinas, São Paulo, Brazil \textbar{}
  pedro.kobori@gmail.com \textbar{}
  \ulink{https://github.com/rokobo}{Github} \textbar{}
  \ulink{https://www.linkedin.com/in/pedrokobori/}{LinkedIn} \textbar{}
  \ulink{https://rokobo.github.io}{Portfolio}
  \par
}

\section*{Skills}
\hrule
\vspace{2mm}
\begin{itemize}[itemsep=0pt]
  \item \textbf{Programming}: Python, SQL, Excel, LaTeX, RegEx, Ruby, Julia, Assembly, SML, Racket, UML.
  \item \textbf{Technologies}: NumPy, Pandas, TensorFlow, Docker, Pytest, GitHub, Spark, Neo4j, Splunk, Knime.
  \item \textbf{Web Development}: Plotly Dash, HTML, CSS, GitHub Actions, Javascript, Matplotlib.
\end{itemize}

\section*{Projects}
\hrule
\vspace{2mm}
\textbf{Big data specialization}, \ulink{https://github.com/rokobo/Big-Data-Specialization}{Project Code}.
\begin{itemize}[itemsep=0pt]
  \item Performed data exploration with Splunk for item market sales performance analysis.
  \item Created a classification model workflow with Knime for assigning labels to high spending users.
  \item Used K-means clustering with Spark (PySpark) for user-segment analysis.
  \item Used chat data modelling with Neo4j for recommending targetted marketing campaigns.
\end{itemize}
\textbf{Productivity autotracker}, \ulink{https://github.com/rokobo/Productivity-autotracker}{Project Code}.
\begin{itemize}[itemsep=0pt]
  \item Implemented activity tracking logic and event categorization in Python to determine user productivity.
  \item Used SQL databases to keep track of user data, trends, settings and milestones.
  \item Made the interface and configuration using Plotly Dash, with easily configurable user settings.
\end{itemize}
\textbf{Machine learning specialization}, \ulink{https://github.com/rokobo/Machine-learning-specialization}{Project Code}.
\begin{itemize}[itemsep=0pt]
  \item Created neural networks for image classification, collaborative filtering for movie recommendation.
  \item Created anomaly detection for server operations, reinforcement learning for lunar lander.
  \item Created clustering for image compression, decision trees for mushroom recognition.
  \item Performed model analysis, data engineering and data visualization to various real-world ML problems.
  \item Technologies: NumPy, TensorFlow, Keras, Scikit-Learn, Matplotlib, Pandas, OpenAI gym.
\end{itemize}
\textbf{Hack computer design and game implementation}, \ulink{https://github.com/rokobo/From-Nand-gates-to-Tetris-implementation}{Project Code}.
\begin{itemize}[itemsep=0pt]
  \item From a NAND gate abstraction and some emulation tools, I implemented the Hack computer software.
  \item Implemented the theoretical hardware architecture using HDL.
  \item Implemented an assembler, to transform assembly files into Hack files (binary code for the Hack platform).
  \item Implemented a translator, translates VM code (stack based operations) into assembly.
  \item Implemented a compiler, to compile Jack files into VM code.
  \item Finally, I implemented a Jack OS and a Jack game.
\end{itemize}
\textbf{DNA sequence analysis using alignment and scoring matrices}, \ulink{https://github.com/rokobo/DNA-Sequence-Analysis}{Project Code}.
\begin{itemize}[itemsep=0pt]
  \item Implemented functions to create alignment and scoring matrices for quantifying similarity in 2 DNA sequences.
  \item Did statistical hypothesis testing with Z-scores of multiple local alignments with one random sequence.
  \item Implemented functions to quantify dissimilarity between 2 strings for spelling check using edit distances.
\end{itemize}
\textbf{Computer network resilience analysis}, \ulink{https://github.com/rokobo/Computer-Network-Resilience-Analysis}{Project Code}.
\begin{itemize}[itemsep=0pt]
  \item Analyzed the connectivity of a computer network by randomly disabling computers on the network.
  \item Compared the resilience of a provided graph and randomly generated ER and UPA graphs (DPA graph variety).
  \item Implemented and compared (time complexity) 3 algorithms for removing computers from the network.
  \item Analyzed how the largest connected component changed depending on the graph creation algorithm.
\end{itemize}

\section*{Education}
\hrule
\vspace{2mm}
\begin{itemize}[itemsep=0pt]
  \item Currently studying by the
        \ulink{https://github.com/ossu/computer-science}{Open Source Society University (OSSU)}
        computer science path. See earned certifications on my portfolio. The OSSU curriculum
        is a complete education in computer science using online materials. It is designed
        according to the curriculum guidelines for undergraduate degree programs in computer science, by IEEE.
  \item Native portuguese, fluent english: Cambridge Advanced English Level 2 Certificate in ESOL International.
\end{itemize}

\end{document}
